\subsubsection*{Messaggi di debug}

Per il debug e\textquotesingle{} possibile impostare il flag D\+E\+B\+UG all\textquotesingle{} interno di \hyperlink{Server_2inc_2Config_8h}{Server/inc/\+Config.\+h} per il server, e Client/int/\+Config.\+h per il client. Questo stampera\textquotesingle{} durante l\textquotesingle{} esecuzione utili messaggi di debug.

\subsubsection*{Wireshark}

Permette di sniffare i pacchetti di rete, una volta avviato si esegue lo sniffing dei pacchetti sull\textquotesingle{} interfaccia di loopback (lo) in caso stiamo testando l\textquotesingle{} applicazione in locale. Per windows è necessario qualche passaggio in più\+:
\begin{DoxyEnumerate}
\item Scaricare \href{http://www.netresec.com/?page=RawCap}{\tt Raw\+Pack}
\item Lanciare in una shell\+: {\ttfamily Raw\+Cap.\+exe -\/f 127.\+0.\+0.\+1 dumpfile.\+pcap}
\item Lanciare in un\textquotesingle{}altra shell\+: {\ttfamily tail -\/c +0 -\/f dumpfile.\+pcap $\vert$ Wireshark.\+exe -\/k -\/i -\/}
\end{DoxyEnumerate}

\subsubsection*{Strace}

Usato per tenere traccia delle chiamate di sistema.

\subsubsection*{Valgrind}

Usato per testare se ci sono dei memory leaks

\subsubsection*{G\+DB}

debugger step by step. Alcuni comandi utili\+:


\begin{DoxyItemize}
\item {\ttfamily set follow-\/fork-\/mode child}\+: Per debuggare il processo figlio di una fork o un pthread
\item {\ttfamily r}\+: per avviare il debug
\item {\ttfamily make} per ricompilare l\textquotesingle{} eseguibile
\item {\ttfamily n}\+: prossima istruzione
\item {\ttfamily b file.\+c\+:$<$number$>$}\+: pr bloccare l\textquotesingle{} esecuzione a una certa lnea di un file, o
\item {\ttfamily b numefunzione}\+: per bloccare l\textquotesingle{} eseucizione ad una certa funzione.
\end{DoxyItemize}

\subsection*{Dr\+Memory}

Simile a Valgrind, ma per windows. 