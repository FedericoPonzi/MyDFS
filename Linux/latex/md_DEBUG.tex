\subsubsection*{Messaggi di debug}

Per il debug e' possibile impostare il flag D\+E\+B\+U\+G all' interno di \hyperlink{Server_2inc_2Config_8h}{Server/inc/\+Config.\+h} per il server, e Client/int/\+Config.\+h per il client. Questo stampera' durante l' esecuzione utili messaggi di debug.

\subsubsection*{Wireshark}

Permette di sniffare i pacchetti di rete, una volta avviato si esegue lo sniffing dei pacchetti sull' interfaccia di loopback (lo) in caso stiamo testando l' applicazione in locale.

\subsubsection*{Valgrind}

Usato per testare se ci sono dei memory leaks

\subsubsection*{G\+D\+B}

debugger step by step. Alcuni comandi utili\+:


\begin{DoxyItemize}
\item {\ttfamily set follow-\/fork-\/mode child}\+: Per debuggare il processo figlio di una fork o un pthread
\item {\ttfamily r}\+: per avviare il debug
\item {\ttfamily make} per ricompilare l' eseguibile
\item {\ttfamily n}\+: prossima istruzione
\item {\ttfamily b \hyperlink{file_8c}{file.\+c}\+:$<$number$>$}\+: pr bloccare l' esecuzione a una certa lnea di un file, o
\item {\ttfamily b numefunzione}\+: per bloccare l' eseucizione ad una certa funzione. 
\end{DoxyItemize}