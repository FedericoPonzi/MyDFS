Il modulo di caching è formato principalmente da tre funzioni\+:

\subsubsection*{1. File di cache}

Una funzione che crea un file di caching temporaneo con nome .nomefile, che viene cancellato una volta chiuso l' ultimo file descriptor associato al file.

\subsubsection*{2. Hit Or Miss}

Una funzione che risponde, prima di una lettura, se il dato che si vuole leggere è presente in cache o meno.

int \hyperlink{Cache_8h_a403785c0f2573dd16d81eff4ab26c20d}{read\+Request(\+My\+D\+F\+S\+Id$\ast$ id, int pos, int size, Cache\+Request$\ast$ req)}

Nel caso in cui il dato che il client vuole leggere non è presente in cache, la funzione ritorna come valore 0 (F\+A\+L\+S\+E) e inserisce all' interno della struttura dato req\+: \begin{DoxyVerb}typedef struct CacheRequest {
    int pos;
    int size;
}CacheRequest;
\end{DoxyVerb}
 La posizione, e la dimensione del primo buco dello spazio che l' utente vuole leggere. Quando questa funzione ritorna true, vuol dire che ho un hit completo e quindi posso procedere con la lettura.

Lo storico delle operazioni di scrittura e lettura viene mantenuto nelle rispettive liste linkate all' interno di \hyperlink{structMyDFSId}{My\+D\+F\+S\+Id}.

\subsection*{3. Write cache}

La write cache è una funzione utilizzata dalla read per scrivere nella cache. Le read non devono sovrascrivere le write effettuate dal client con i dati ricevuti dal server, quindi si usa questa funzione di supporto per scrivere nella cache senza andare a sovrascrivere zone di memoria protette. 