Il file di configurazione contiene le impostazioni per il server. Le righe che iniziano con un simbolo {\ttfamily \#} sono considerate commenti. Le impostazioni necessarie sono\+:


\begin{DoxyItemize}
\item {\ttfamily N\+U\+M\+B\+E\+R\+\_\+\+O\+F\+\_\+\+C\+O\+N\+N\+E\+C\+T\+I\+O\+N $<$number$>$}\+: Equivale al numero di thread o processi da spawnare. Ogni processo/thread gestisce una connessione (=mydfs\+\_\+open)
\item {\ttfamily P\+R\+O\+C\+E\+S\+S\+\_\+\+O\+R\+\_\+\+T\+H\+R\+E\+A\+D $<$number$>$}\+: 1 per i processi, 0 per i threads.
\item {\ttfamily P\+O\+R\+T\+\_\+\+N\+U\+M\+B\+E\+R 9001}\+: Numero di porta su cui il server si mette in ascolto.
\item {\ttfamily R\+O\+O\+T\+\_\+\+P\+A\+T\+H /tmp/}\+: Path del file system su cui deve lavorare il server.
\end{DoxyItemize}

\subsubsection*{Esempio di file di configurazione}

\begin{DoxyVerb}##FILE DI CONFIG

###
##Numero massimo di connessioni contemporanee:
###
NUMBER_OF_CONNECTION 2

###
## 1 per usare i processi, 0 per usare i threads
###
PROCESS_OR_THREAD 0

###
## Numero di porta su cui ascoltare la connessione:
###
PORT_NUMBER 9001

###
## Root path file
###
ROOT_PATH /tmp/\end{DoxyVerb}
 